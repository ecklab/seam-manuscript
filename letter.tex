\documentclass[11pt]{article}


\usepackage{tikz}
\usepackage{geometry}
\usepackage{graphicx}
\usepackage{color}
\usepackage[export]{adjustbox}


\setlength{\parindent}{0em}
\setlength{\parskip}{1em}


\usepackage{geometry}
\geometry{margin=1in}

\renewcommand*\arraystretch{1.05}

\begin{document}

\footnotesize
%\color{lightblue}
\begin{tikzpicture}
  \node at (0,0) {\includegraphics[height=1.1in, keepaspectratio = true]{Block_I}};
  \node at (13.4, -0.05) {\vspace{1cm}\begin{tabular}{l}
      \\ \sc Daniel J. Eck \\
       \it Department of Statistics \\
       University of Illinois \\
	   Computing Applications Building, Room 152 \\
	   605 E. Springfield Ave. \\
	   Champaign, IL 61820 \\
       \texttt{dje13@illinois.edu} \\ 
       \end{tabular} };
\end{tikzpicture}

\normalsize 


Dear Editor: 

We are submitting our manuscript ``SEAM methodology for context-rich player matchup evaluations in baseball'' for your consideration at the American Statistician. This manuscript is at the interface between statistical methodology and a current baseball industry problem involving the estimation and prediction of batted-ball locations. 

In this paper we develop the SEAM method as a technique for estimated the batted-ball distribution for a matchup involving a specific batter and pitcher. Our method overcomes the challenge of sparse matchup data by pooling in batted-ball locations of other players that are weighted by similarity. Mathematical justification for this method is provided. A detailed validation study reveals that our SEAM approach more accurately predicts the locations of batted-ball locations than existing publicly available methods.

This manuscript has not been published elsewhere, and it has not been submitted for publication elsewhere.

Thank you for your consideration.

Sincerely,

Julia Wapner \\
David Dalpiaz \\
Daniel J. Eck


\end{document}